Energy efficiency is a topic that had garnered much attention lately. With an increasing need for electrical apparatus, saving as much energy as possible is a necessity for allowing our current technological development to continue. Energy can be saved in all aspects of a micro-controller, and every bit counts. However, as hardware components and architectures have been perfected over many years, there is very little to gain here. The system architecture allows for a moderate amount of energy saving, but the most important part is the types of algorithms being used, as the difference between a good and a bad one can be humongous.

Producing customized computer chips for each new development project is expensive and not very flexible. Hence, most start by developing on so called development cards which are specially made for the job by featuring nearly all thinkable features and pins for connecting different components. After development have finished, a computer chip with all unnecessary components removed can then be produced.

The STK1000 is one such development card and will be used for this assignment. It features an AT32AP7000 microcontroller from Atmel containing an AVR32 processor core. AVR32 is a 32-bit RISC processor architecture made especially for use in embedded systems that demands a relatively high amount of performance. It has a high clock frequency compared to those used in many other microcontrollers and features a seven stage pipeline which allows for up to three different instructions to be executed at the same time. The microcontroller also makes use of memory mapped IO which means that interaction with external devices is done through simple memory read and write operations.

A phenomenon that has become a problem with increasing processing speed is bouncing. When a mechanical switch is toggled, the impact between the switch and the holder can cause the button to bounce a little which causes contact to be lost and regained. This has not been a problem in the past because hardware has been too slow to detect it, but modern microcontrollers are so fast that they can react to the event before the switch has stabilized it’s position, which can cause multiple events to be detected. To compensate for this, a technique called debouncing is used, which essentially pauses execution for a short time to allow the switch to stabilize.

With all programs comes the need for debugging, unless the programming is flawless, which of course is highly unlikely. Most of the currently available microcontrollers therefore have a built-in debugging module to make this easy. This module allows for step-by-step execution of the program, information about what is stored where at every time and more.

The purpose of this assignment was to create an assembly program for the STK1000 that would turn on the central LED, and then allow a user to move the light left and right with the buttons on the board. Furthermore, it was required to read the button states in an interrupt routine, while processing them in a main loop that would sleep between events.
